%%
%% Class homework & solution template for latex
%% Alex Ihler
%%
\documentclass[twoside,11pt]{article}
\usepackage{amsmath,amsfonts,amssymb,amsthm}
\usepackage{graphicx,color}
\usepackage{verbatim,url}
\usepackage{listings}
\usepackage{upquote}
\usepackage[T1]{fontenc}
%\usepackage{lmodern}
\usepackage[scaled]{beramono}
%\usepackage{textcomp}

% Directories for other source files and images
\newcommand{\bibtexdir}{../bib}
\newcommand{\figdir}{fig}

\newcommand{\E}{\mathrm{E}}
\newcommand{\Var}{\mathrm{Var}}
\newcommand{\N}{\mathcal{N}}
\newcommand{\matlab}{{\sc Matlab}\ }

\setlength{\textheight}{9in} \setlength{\textwidth}{6.5in}
\setlength{\oddsidemargin}{-.25in}  % Centers text.
\setlength{\evensidemargin}{-.25in} %
\setlength{\topmargin}{0in} %
\setlength{\headheight}{0in} %
\setlength{\headsep}{0in} %

\renewcommand{\labelenumi}{(\alph{enumi})}
\renewcommand{\labelenumii}{(\arabic{enumii})}

\theoremstyle{definition}
\newtheorem{MatEx}{M{\scriptsize{ATLAB}} Usage Example}

\definecolor{comments}{rgb}{0,.5,0}
\definecolor{backgnd}{rgb}{.95,.95,.95}
\definecolor{string}{rgb}{.2,.2,.2}
\lstset{language=Matlab}
\lstset{basicstyle=\small\ttfamily,
        mathescape=true,
        emptylines=1, showlines=true,
        backgroundcolor=\color{backgnd},
        commentstyle=\color{comments}\ttfamily, %\rmfamily,
        stringstyle=\color{string}\ttfamily,
        keywordstyle=\ttfamily, %\normalfont,
        showstringspaces=false}
\newcommand{\matp}{\mathbf{\gg}}




\begin{document}

\centerline{\Large Homework 3}
\centerline{Zachary DeStefano, 15247592}
\centerline{CS 273A: Winter 2015}
\centerline{\bf Due: January 27, 2015}

\section*{Problem 1}

\subsection*{Part a}

This is the plot of class 0 versus class 1, which is linearly separable. \\
\begin{figure}[h]
\centering
\includegraphics[width=4 in]{prob1aPlot1.png}
\caption{Class 0 and Class 1 Points}
\end{figure}
\newpage
This is the plot of class 1 versus class 2, which is not linearly separable. \\
\begin{figure}[h]
\centering
\includegraphics[width=4 in]{prob1aPlot2.png}
\caption{Class 1 and Class 2 Points}
\end{figure}
\\
Here is the code to complete part a
\lstinputlisting[firstline=1, lastline=22]{prob1.m}
\newpage
\subsection*{Part b}

Here is the plot of class 0 and class 1 along with the specified decision boundary\\
\\
\begin{figure}[h]
\centering
\includegraphics[width=4 in]{prob1bPlot1.png}
\caption{Class 0 and Class 1 points along with the sample decision boundary}
\end{figure}
\newpage
Here is the plot of class 1 and class 2 along with that decision boundary\\
\\
\begin{figure}[h]
\centering
\includegraphics[width=4 in]{prob1bPlot2.png}
\caption{Class 1 and Class 2 points along with the sample decision boundary}
\end{figure}
\\
Here is the code for plot2DLinear
\lstinputlisting[firstline=1, lastline=21]{@logisticClassify2/plot2DLinear.m}
\newpage
Here is the code to complete part b. It does rely on some of the variables from the part a code.
\lstinputlisting[firstline=24, lastline=46]{prob1.m}

\subsection*{Part c}

I got an error rate of $0.0505$ for data set A\\
 and an error rate of $0.4646$ for data set B.\\
\\
Here is the code for predict.m
\lstinputlisting[firstline=1, lastline=20]{@logisticClassify2/predict.m}
\newpage
Here is the code to compute the error rates for part C. It relies on the code for Part B.
\lstinputlisting[firstline=46, lastline=57]{prob1.m}

\subsection*{Part d}

An individual derivative is as follows:
\[
J_j'(\theta_i) = (1-y^{(j)})\theta_i x_i^{(j)} - \frac{exp(-z)\theta_i x_i^{(j)}}{1+exp(-z)} + 2 \alpha \theta_i
\]

\section*{Problem 2}

\subsection*{Part a}

It can separate the points in a,b, and c, but you can arrange the points in d such that this won't separate them. The VC dimension is thus 3.

\subsection*{Part b}

It can separate the points in a,b,c, and d, so the VC dimension is 4.

\end{document}
