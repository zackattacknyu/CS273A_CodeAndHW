%%
%% Class homework & solution template for latex
%% Alex Ihler
%%
\documentclass[twoside,11pt]{article}
\usepackage{amsmath,amsfonts,amssymb,amsthm}
\usepackage{graphicx,color}
\usepackage{verbatim,url}
\usepackage{listings}
\usepackage{upquote}
\usepackage[T1]{fontenc}
%\usepackage{lmodern}
\usepackage[scaled]{beramono}
%\usepackage{textcomp}

% Directories for other source files and images
\newcommand{\bibtexdir}{../bib}
\newcommand{\figdir}{fig}

\newcommand{\E}{\mathrm{E}}
\newcommand{\Var}{\mathrm{Var}}
\newcommand{\N}{\mathcal{N}}
\newcommand{\matlab}{{\sc Matlab}\ }

\setlength{\textheight}{9in} \setlength{\textwidth}{6.5in}
\setlength{\oddsidemargin}{-.25in}  % Centers text.
\setlength{\evensidemargin}{-.25in} %
\setlength{\topmargin}{0in} %
\setlength{\headheight}{0in} %
\setlength{\headsep}{0in} %

\renewcommand{\labelenumi}{(\alph{enumi})}
\renewcommand{\labelenumii}{(\arabic{enumii})}

\theoremstyle{definition}
\newtheorem{MatEx}{M{\scriptsize{ATLAB}} Usage Example}

\definecolor{comments}{rgb}{0,.5,0}
\definecolor{backgnd}{rgb}{.95,.95,.95}
\definecolor{string}{rgb}{.2,.2,.2}
\lstset{language=Matlab}
\lstset{basicstyle=\small\ttfamily,
        mathescape=true,
        emptylines=1, showlines=true,
        backgroundcolor=\color{backgnd},
        commentstyle=\color{comments}\ttfamily, %\rmfamily,
        stringstyle=\color{string}\ttfamily,
        keywordstyle=\ttfamily, %\normalfont,
        showstringspaces=false}
\newcommand{\matp}{\mathbf{\gg}}




\begin{document}

\centerline{\Large Homework 4}
\centerline{Zachary DeStefano, 15247592}
\centerline{CS 273A: Winter 2015}
\centerline{\bf Due: February 24, 2015}

\section*{Problem 1}

\subsection*{Part a}

This is the plot of class 0 versus class 1 using the SVM solver.\\
The value of b was -17.2697\\
The w vector was $(6.3572,-5.3693)$\\

\begin{figure}[h]
\centering
\includegraphics[width=6 in]{prob1Plot1.png}
\caption{Classification Plot with support vectors starred in yellow}
\end{figure}
\newpage
Here is the code to obtain the previous plot. The details of how I transformed into a version compatible with quadprog are in the comments. 
\lstinputlisting[firstline=1, lastline=49]{prob1.m}

\newpage

This is the code to obtain $\alpha$, verify it, and then plot the classification bounday. 
\lstinputlisting[firstline=51, lastline=119]{prob1.m}

\newpage

\section*{Problem 2}

\subsection*{Part a}

To calculate the entropy, I did the following
\[
H(y) = p(y=1)log_2(\frac{1}{p(y=1)}) + p(y=-1)log_2(\frac{1}{p(y=-1)})
\]
The entropy ends up being $0.9743$
\\

\subsection*{Part b}

You should split on feature 2 first. Here is the information gain for all the variables:\\
\\
\begin{tabular}{ c | c }
  Feature & Information Gain\\
  \hline                       
  1 & 0.0245 \\
  2 & 0.5059 \\
  3 & -0.0097 \\
  4 & 0.0930 \\
  5 & -0.0095 \\      
\end{tabular}

\newpage

\subsection*{Code for a and b}

This is the code to complete Part A and B. The code for the getEntropy function will be shown later.

\lstinputlisting[firstline=1, lastline=48]{prob2.m}

\subsection*{Part C}

Here is the decision tree for Part C

\begin{figure}[h]
\centering
\includegraphics[width=6 in]{hw4prob2DecisionTree.png}
\caption{Decision Tree}
\end{figure}

\newpage

Here is the code used in Part C. It relies on the data set up from Part A.

\lstinputlisting[firstline=50, lastline=82]{prob2.m}

\subsection*{Matlab functions written for Problem 2}

This is the code for the getEntropy function
\lstinputlisting[firstline=1, lastline=15]{getEntropy.m}

This is the code for the getDecTreeSplit function
\lstinputlisting[firstline=1, lastline=52]{getDecTreeSplit.m}


\section*{Problem 3}

\subsection*{Part a}

The validation MSE that I obtained was 0.7133. \\
This is the code I used to obtain it:
\lstinputlisting[firstline=1, lastline=11]{prob3.m}

\newpage

\subsection*{Part b}

At the maxDepth parameter increases, the model grows increasingly complex. When we plot validation MSE versus maxDepth, it is clear that it does begin to overfit. The lowest validation MSE is obtained when maxDepth is 8 and after that overfitting seems to begin. The validation MSE at 8 ends up being 0.4355. Here is the plot which illustrates that below.

\begin{figure}[h]
\centering
\includegraphics[width=6 in]{prob3plot1.png}
\caption{Validation MSEs for the Decision Trees}
\end{figure}

\newpage

Here is the code used for Part b\\
\lstinputlisting[firstline=13, lastline=29]{prob3.m}


\end{document}
