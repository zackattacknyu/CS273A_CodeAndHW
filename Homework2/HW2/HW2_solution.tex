%%
%% Class homework & solution template for latex
%% Alex Ihler
%%
\documentclass[twoside,11pt]{article}
\usepackage{amsmath,amsfonts,amssymb,amsthm}
\usepackage{graphicx,color}
\usepackage{verbatim,url}
\usepackage{listings}
\usepackage{upquote}
\usepackage[T1]{fontenc}
%\usepackage{lmodern}
\usepackage[scaled]{beramono}
%\usepackage{textcomp}

% Directories for other source files and images
\newcommand{\bibtexdir}{../bib}
\newcommand{\figdir}{fig}

\newcommand{\E}{\mathrm{E}}
\newcommand{\Var}{\mathrm{Var}}
\newcommand{\N}{\mathcal{N}}
\newcommand{\matlab}{{\sc Matlab}\ }

\setlength{\textheight}{9in} \setlength{\textwidth}{6.5in}
\setlength{\oddsidemargin}{-.25in}  % Centers text.
\setlength{\evensidemargin}{-.25in} %
\setlength{\topmargin}{0in} %
\setlength{\headheight}{0in} %
\setlength{\headsep}{0in} %

\renewcommand{\labelenumi}{(\alph{enumi})}
\renewcommand{\labelenumii}{(\arabic{enumii})}

\theoremstyle{definition}
\newtheorem{MatEx}{M{\scriptsize{ATLAB}} Usage Example}

\definecolor{comments}{rgb}{0,.5,0}
\definecolor{backgnd}{rgb}{.95,.95,.95}
\definecolor{string}{rgb}{.2,.2,.2}
\lstset{language=Matlab}
\lstset{basicstyle=\small\ttfamily,
        mathescape=true,
        emptylines=1, showlines=true,
        backgroundcolor=\color{backgnd},
        commentstyle=\color{comments}\ttfamily, %\rmfamily,
        stringstyle=\color{string}\ttfamily,
        keywordstyle=\ttfamily, %\normalfont,
        showstringspaces=false}
\newcommand{\matp}{\mathbf{\gg}}




\begin{document}

\centerline{\Large Homework 2}
\centerline{Zachary DeStefano, 15247592}
\centerline{CS 273A: Winter 2015}
\centerline{\bf Due: January 20, 2015}

\section*{Problem 1}

\subsection*{Part a}

Here is the code to complete part a
\lstinputlisting[firstline=2, lastline=8]{prob1.m}

\subsection*{Part b}

Here is the code to complete part b. It does rely on the code from part a:
\lstinputlisting[firstline=9, lastline=25]{prob1.m}
The MSE for the training data was 1.1277\\
The MSE for the test data was 2.2423\\

\newpage

Here is the plot for part b:
\begin{figure}[h]
\centering
\includegraphics[width=5 in]{prob1bPlot.png}
\caption{The training data, test data, and the predicted values}
\end{figure}

\newpage

\subsection*{Part c}

%	For direct matlab code
%\begin{lstlisting}
%\end{lstlisting}

Here are the plots of the $f(x)$ functions and the training and test data. The legend is the same as the plot in part b. Each plot represents a prediction for a different polynomial degree.\\
\begin{figure}[h]
\centering
\includegraphics[width=\columnwidth]{prob1cPlotA1.png}
\end{figure}
\begin{figure}[h]
\centering
\includegraphics[width=\columnwidth]{prob1cPlotA2.png}
\end{figure}

\newpage

\begin{figure}[h]
\centering
\includegraphics[width=\columnwidth]{prob1cPlotA3.png}
\end{figure}

\newpage

Here is the plot of the training and test error. Based on this plot, I would choose a polynomial of degree 10.
\begin{figure}[h]
\centering
\includegraphics[width=\columnwidth]{prob1cPlotB.png}
\end{figure}

\newpage
This is the code used to accomplish these plots. It is a continuation of the code from part a as it uses the arrays created there. 
\lstinputlisting[firstline=30, lastline=77]{prob1.m}

\newpage

\section*{Problem 2}

This is the plot. As can be observed, the minimum average MSE occurs where $degree=5$. \\
\begin{figure}[h]
\centering
\includegraphics[width=\columnwidth]{prob2Plot.png}
\caption{The average cross-validation MSE as function of polynomial degree}
\end{figure}
\\
The cross-validation error with degree 5 was $0.5911$. \\
The error for the degree 5 polynomial on actual test data was $1.0344$ in the previous problem. \\
Therefore using cross-validation reduced the error that we saw.\\
\\
It also changed which degree polynomial had the lowest error. On the actual test data, the degree 10 polynomial had the lowest error with an average MSE of $0.6091$.
\\
\newpage
Here is the code that I used to get the numbers and the plot for problem 3
\lstinputlisting[firstline=1, lastline=40]{prob2.m}


\end{document}
